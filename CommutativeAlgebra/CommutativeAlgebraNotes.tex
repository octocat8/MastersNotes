\documentclass[12pt]{book}
\date{\today}
\title{Commutative Algebra: G0A82a}
\author{Lael John}
\usepackage{amsmath, amsfonts, amssymb, amsthm}

\newtheorem{theorem}{Theorem}[chapter]
\newtheorem{lemma}[theorem]{Lemma}
\newtheorem{corollary}{Corollary}[theorem]
\newtheorem{proposition}{Proposition}[section]

\theoremstyle{definition}
\newtheorem*{definition}{Definition}
\newtheorem{example}{Example}[chapter]
\newtheorem*{huh}{Thoughts}
\newtheorem*{remark}{Remark}
\begin{document}
\maketitle
\chapter*{Preface}
These notes for the course on commutative algebra are to serve as a reference point for all future work in algebraic geometry, and the larger realm of algebra in general. From what I can gather, we begin with a basic overview of ring theory, and then proceed to study modules and tensor products, before finally arriving at an introduction to category theory.
\tableofcontents
\chapter{Rings}
We begin by going over what we already know about rings, and use this knowledge as a foundation to pursue further advanced algebraic structures.
\section{Definitions}
\begin{definition}
    A \textit{ring} is an algebraic structure $(R, +, .)$, where under the operation $+, (R, +)$ forms an abelian group, and under the operation $., (R, .)$ forms a semigroup (only closed and associative). 
\end{definition}
\begin{remark}
    Here we call the ring \textit{commutative}, if the $.$ operation is abelian. Similarly we can also talk about the existence of a \textit{unity} in the ring for that same operation. NOTE: the only ring where the unity is not distinct from the identity (used to denote additive identity) is the $0$ ring, or the ring $(\{0\}, + , .)$. This ring is generally treated as an edge case, even though trivially, it is a field (a commutative division ring). For most rings that we talk about in the following sections, it is assumed that they are commutative (otherwise the course name would be redundant, and that they all have a unity.)
\end{remark}
\begin{definition}
    A \textit{homomorphism} $f: R \to R$ is a mapping from one ring to another satisfying the following properties \begin{enumerate}
        \item $f(x + y) = f(x) + f(y)$, which basically means that the images in the target ring share the same structure under addition.
        \item $f(xy) = f(x)f(y)$, again, meaning that the images in the target share the same structure under multiplication.
        \item Finally, $f(1_R) = 1_S$. The unity from one ring, under a homomorphism, must be mapped to the unity in the other.
    \end{enumerate}
\end{definition}
\begin{remark}
    A subring is a subset of a ring that has a ring structure with respect to the parent operations. A homomorphism exists between such a subring and its parent. Homomorphisms are basically functions, and therefore when we consider their composition, the resulting function is also a homomorphism.
\end{remark}
\begin{definition}
    An \textit{IDEAL} $I$ of a ring $R$ is a subring of $R$ that is also has the following property, namely that $RI \subset I$ or equivalently, $$\forall r \in R, i \in I, ri \in I$$ With this sort of subring, we can then begin to talk about $R/I$ or the quotient ring, i.e. $\{ x + I | x \in R \}$ with operations of $+, .$ defined slightly differently on this ring. Given an ideal, and such a quotient ring generated by it, one can find a canonical homomorphism $\pi: R \to R/I, \pi(x) \mapsto x + I$.
\end{definition}
\begin{huh}
    Consider a ring $R$ and an ideal $I$. Considering the set of ideals in $R$ containing $I$, they are also ideals, say $\{I_a\}_{a \in A}$. Also now considering the set of ideals in the quotient ring $R/I$, we see that they are subrings such that $(x + I)\mathcal{I} \subset \mathcal{I}$
\end{huh}
\begin{definition}
    When we talk about a ring homomorphism (ring map in future), we can think about the elements in the domain that map to the identity in the target, i.e. $\{ x \in R| f(x) = 0\}$. This set is so important in our study, we term it ker $F$ or the \textit{kernel} of $f$, denoted by $f^{-1} (0)$. In a similar vein, the image of $f$, denoted im $f = f(R)$
\end{definition}
\end{document}