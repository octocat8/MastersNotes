\documentclass[12pt]{book}
\date{\today}
\title{Commutative Algebra: G0A82a}
\author{Lael John}
\usepackage{amsmath, amsfonts, amssymb, amsthm}

\newtheorem{theorem}{Theorem}[chapter]
\newtheorem{lemma}[theorem]{Lemma}
\newtheorem{corollary}{Corollary}[theorem]
\newtheorem{proposition}{Proposition}[section]

\theoremstyle{definition}
\newtheorem*{definition}{Definition}
\newtheorem{example}{Example}[chapter]
\newtheorem*{huh}{Thoughts}
\newtheorem*{remark}{Remark}
\begin{document}
\maketitle
\chapter*{Preface}
These notes for the course on commutative algebra are to serve as a reference point for all future work in algebraic geometry, and the larger realm of algebra in general. From what I can gather, we begin with a basic overview of ring theory, and then proceed to study modules and tensor products, before finally arriving at an introduction to category theory.
\tableofcontents
\chapter{Rings}
We begin by going over what we already know about rings, and use this knowledge as a foundation to pursue further advanced algebraic structures.
\section{Definitions}
\begin{definition}
    A \textit{ring} is an algebraic structure $(R, +, .)$, where under the operation $+, (R, +)$ forms an abelian group, and under the operation $., (R, .)$ forms a semigroup (only closed and associative). 
\end{definition}
\begin{remark}
    Here we call the ring \textit{commutative}, if the $.$ operation is abelian. Similarly we can also talk about the existence of a \textit{unity} in the ring for that same operation. NOTE: the only ring where the unity is not distinct from the identity (used to denote additive identity) is the $0$ ring, or the ring $(\{0\}, + , .)$. This ring is generally treated as an edge case, even though trivially, it is a field (a commutative division ring). For most rings that we talk about in the following sections, it is assumed that they are commutative (otherwise the course name would be redundant, and that they all have a unity.)
\end{remark}
\begin{definition}
    A \textit{homomorphism} $f: R \to R$ is a mapping from one ring to another satisfying the following properties \begin{enumerate}
        \item $f(x + y) = f(x) + f(y)$, which basically means that the images in the target ring share the same structure under addition.
        \item $f(xy) = f(x)f(y)$, again, meaning that the images in the target share the same structure under multiplication.
        \item Finally, $f(1_R) = 1_S$. The unity from one ring, under a homomorphism, must be mapped to the unity in the other.
    \end{enumerate}
\end{definition}
\begin{remark}
    A subring is a subset of a ring that has a ring structure with respect to the parent operations. A homomorphism exists between such a subring and its parent. Homomorphisms are basically functions, and therefore when we consider their composition, the resulting function is also a homomorphism.
\end{remark}
\begin{definition}
    An \textit{IDEAL} $I$ of a ring $R$ is a subring of $R$ that is also has the following property, namely that $RI \subset I$ or equivalently, $$\forall r \in R, i \in I, ri \in I$$ With this sort of subring, we can then begin to talk about $R/I$ or the quotient ring, i.e. $\{ x + I | x \in R \}$ with operations of $+, .$ defined slightly differently on this ring. Given an ideal, and such a quotient ring generated by it, one can find a canonical homomorphism $\pi: R \to R/I, \pi(x) \mapsto x + I$.
\end{definition}
\begin{huh}
    Consider a ring $R$ and an ideal $I$. Considering the set of ideals in $R$ containing $I$, they are also ideals, say $\{I_a\}_{a \in A}$. Also now considering the set of ideals in the quotient ring $R/I$, we see that they are subrings such that $(x + I)\mathcal{I} \subset \mathcal{I}$
\end{huh}
\begin{definition}
    When we talk about a ring homomorphism (ring map in future), we can think about the elements in the domain that map to the identity in the target, i.e. $\{ x \in R| f(x) = 0\}$. This set is so important in our study, we term it ker $F$ or the \textit{kernel} of $f$, denoted by $f^{-1} (0)$. In a similar vein, the image of $f$, denoted im $f = f(R)$
\end{definition}
\begin{huh}
    Now when we consider a ring map, the kernel of $f$ is an ideal of the domain ring, (trivial to prove). However, when we consider the image of the map, in $R_2$, then we cannot directly say that it is an ideal, merely that it is a subring (because it does map all of $R_1$'s structure into $R_2$). Now that we have those two observations, when thinking about the quotient ring of $R_1/\textnormal{ker} f$, we know that all the elements are of the form $r + \textnormal{ker} f$. This new quotient ring is apparently isomorphic to the image of $f$. Which should be easy enough to see, a well defined function $\phi: R_1/\textnormal{ker} f \to \textnormal{im} f$ where each element in the quotient ring is mapped to its corresponding image. i.e. $\phi(r + \textnormal{ker} f) = f(r)$.
    Now clearly we can see that $\phi$ is a homomorphism, and it remains to be proved as to whether it is a bijection. Clearly, it is so. Thus we gain an understanding of the first ring isomorphism theorem.
\end{huh}
\begin{definition}
    Pivoting slightly, we define a \textit{zero-divisor} in $R$ to be a non-zero element such that, when operated upon with another non-zero element, it results in the identity. $xy = 0; x, y \neq 0$. Then both are zero divisors. If a ring has no zero divisors, it is called an \textit{Integral domain.} A related term \textit{nilpotent}, refers to an element $x \in R$ such that $\exists n \in \mathbb{N}; x^n = 0$. Finally, units in $R$ are those elements that have inverses.
\end{definition}
We now discuss the types of ideals, and see where each of those definitions leads us in our understanding of rings.
\begin{definition}
    A \textit{principal ideal}, is one that can be generated by a single element within it. $I \subset R; I = <x>, x \in R$. (Very clearly we can see that $R$ itself is a principal ideal, generated by $1$.)
\end{definition}
\begin{proposition}
    The folowing are equivalent \begin{enumerate}
        \item $R$ is a field
        \item $R$ has only the trivial ideals
        \item $\forall f: R \to R'$, $f$ is either injective or im $f = 0$.
    \end{enumerate}
\end{proposition}
\begin{proof}
    We go from 1 to 2, 2 to 3 and 3 to 1.
    If we assume $R$ is a field, then we can immediately see that it contains the ideals generated by $1$ and $0$. Now for any other ideal that could possibly exist (say $I$), there is at least one non-zero, non-unity element in it. BUT because $R$ is a field, when multiplying with elements from $R$, since $I$ is an ideal we get $1 \in I \implies I = R$.\\
    IF we assume $R$ has only trivial ideals, and consider a ring map to $R'$, then since ker $f$ needs to be an ideal, thus it must be either $<0>$ or $<1>$. But if it is the former, then it only maps the zero of one ring to the zero in the other, hence the function is injective (characterisation). If the latter, then all elements are mapped to the zero of the other ring.
\end{proof}
\begin{definition}
    A \textit{prime} ideal in $R$ is one such that for any product of 2 elements that lies within the ideal, one of them must be originally part of the ideal. (similar to how we define prime numbers dividing products)
\end{definition}
\begin{definition}
    A \textit{maximal} ideal is one which does not have any ideal larger than it, other than the ring itself. 
\end{definition}
\begin{example}
    Here, a classic example. if $K$ is a field, and $K[x_1, x_2 \ldots x_n]$ the ring of polynomials generated, then $(x_1 - a_1)(x_2 - a_2)\ldots(x_n - a_n)$ generates a maximal ideal.
\end{example}
\begin{proposition}
    A ring map preserves the property of primeness of ideals, but not the property of maximalness.
\end{proposition}
\begin{remark}
    The proof that every ring has at least 1 maximal ideal stems from using Zorn's lemma, which is as follows: Given any non-empty partially ordered set such that every chain has an upper bound, such a set is guaranteed to have maximal element.
\end{remark}
We embark on our last two definitions:
\begin{definition}
    $R$ is a \textit{local} ring, if it has a unique maximal ideal. In particular, if we consider such a local ring, then the quotient ring generated by such a maximal ideal is called the \textit{residue field}
\end{definition}
\section{Operations on Ideals}.
Now that we've built up the groundwork and background regarding rings and their ideals, we consider the operations possible when dealing with two or more ideals. 
This section should be relatively short, but may cover some interesting edge cases.
\begin{definition}
    The \textit{sum} of two ideals is defined to be the resulting set of all the sums of elements within each. As expected, such a sum is also guaranteed to be an ideal. This ideal is trivially also the smallest one containing the two component ideals. In fact, that is true for all finite sums. However once we deal with infinite families of ideals (yes such exist/must be taken into account) we define the sume $\sum_{a \in A} I_a$ as $\sum_{a \in A} x_a$ where finitely many $x_a$'s are non-zero.
\end{definition}
Trivially, intersection is also an operation on ideals that results in an ideal (both in the finite case, and infinite case).
\begin{definition}
    We consider the \textit{product} of two ideals to be the set \textbf{generated} by all the individual products formed pairwise between the two. Naturally, this set is also an ideal (Think about it for a minute.) This definition of a product can be used to think about the n-th power of an ideal, (for a positive integer n). This is then the set generated by all the pairwise multiplications of elements in the ideal. (consider this like a $\binom{n}{2}$ set of generators.) 
\end{definition}
\begin{definition}
    Two ideals are \textit{coprime} if their sum is the ring itself.
\end{definition}
\begin{remark}
    If $I, J$ coprime, then $I \cap J = (I + J)(I \cap J) = I(I + J) \cap J(I + J) \subset IJ$
\end{remark}
Now in building up all this theory, as an interesting side note, we can prove the Chinese remainder theorem (which states that given a set of congruences 3 or more if I remember correctly,$x \cong a_i \textnormal{mod} n_i $ each saying  then there existed a unique solution mod $n_1n_2\ldots$, where each $n_i$ is pairwise coprime). Our proof hinges on this co-prime assumption, constructing a ring map between the ring and n quotient rings, each modulo ideals that taken pairwise are coprime. \\
We finally move on to our last two operations, 
\begin{definition}
    The \textit{ideal quotient} is defined as being $$(I : J) = \{x \in R| xJ \subset I\}$$. (confusing, requires some thought.)
    In particular if we take the quotient $(0: J)$, that is the same as the annihilator of $J$.
\end{definition}
\begin{definition}
    Let $R$ be a ring and $I$ an ideal. Then the \textit{radical} of $I$, denoted by $\sqrt{I} = \{x \in R| x^n \in I, n \in \mathbb{N}\}$. A radical ideal therefore, is an ideal $I$, where $I = \sqrt{I}$
\end{definition}
\end{document}