\documentclass[12pt]{book}
\date{\today}
\title{Differential Geometry 1: G0B08a}
\author{Lael John}
\usepackage{amsmath, amsfonts, amssymb, amsthm}

\newtheorem{theorem}{Theorem}[chapter]
\newtheorem{lemma}[theorem]{Lemma}
\newtheorem{corollary}{Corollary}[theorem]
\newtheorem{proposition}{Proposition}[section]

\theoremstyle{definition}
\newtheorem*{definition}{Definition}
\newtheorem{example}{Example}[chapter]
\newtheorem*{huh}{Thoughts}
\newtheorem*{remark}{Remark}
\begin{document}
\maketitle
\chapter*{Preface}
This course seems to involve playing around with manifolds (which I'm given to understand are surfaces not situated \textit{within} another space, but rather are the space of study themselves). In particular, this course seems to involve a lot of geometry (DUH) and building up a calculus on various weird shapes (so to speak). Over the course of the semester, we will see how my initial impressions change.
\tableofcontents
\chapter{Differentiable Manifolds}
This chapter is supposed to cover all the basics that I will end up using this semester in my study of these strange things called \textit{manifolds}, so we build up the required theory, starting from topological spaces, then on to topological manifolds, and then mapping, charting and atlas-ing our way to differentiable manifolds.
\section{Topological Spaces}
This just rehashes now what we've seen in all previous lectures (and is supposed to be prerequisite material) but once more unto the breach
\begin{definition}
    A \textit{topological space} $(X, \tau)$ is when you are given a set $X$ and $\tau \subset P(X)$ such that the following axioms hold true \begin{enumerate}
        \item $\phi, X \in \tau$
        \item For any set $\{U_i\}_{i \in I}$ of $U_i \in \tau \forall i \in I$, then $$ \bigcup_{i \in I} U_i \in \tau$$ where $I$ is an arbitrary index set. 
        \item For a set of $\{U_i\}_{i = 1}^n$, again, where each element is in $\tau$, then $$\bigcap_{i = 1}^n U_i \in \tau$$
    \end{enumerate}
    The elements of $\tau$ are called \textit{open} sets. 
    \begin{huh}Note that you don't really need to define an explicit "finite intersection closure", you could just work with intersection, and then proceed by induction to prove that it works for a finite number of sets.\end{huh}
\end{definition}
\begin{definition}
    Let $p \in X$. Then a \textit{neighborhood} of $p$ is an open subset $U \in \tau$ such that $p \in U \in \tau$
\end{definition}
Finally we think about subsets in a topological space, and whether these arbitrary subsets themselves have a topological structure. It turns out that they do.
\begin{definition}
    If $Y \subset X$, then $(Y, \tau_Y)$ is a topological space where $$\tau_Y = \{ U \cap Y | U \in \tau\}$$ This topology is called an induced topology (intersection of the parents open sets with the subset under scrutiny.)
\end{definition}
We know pause and recall a certain fact about equivalence relations partition the sets they are used on.
\begin{definition}
    Let $\sim$ be an equivalence relation on a topological space $(X, \tau)$ (going to ommit writing the $\tau$ in future). Now consider $\pi: X \to X/\sim$ being a function. Here, since we have a function from a set to it's quotient set (partitioned into equivalence classes by an equivalence relation), we can say $X/\sim$ is a topologial space, by saying $U \in X/\sim$ open whenever $\pi^{-1}(U)$ open in $X$. This definition also implicitly makes $\pi$ a continuous function (because by definition, it pulls back open sets in the target to open sets in the domain.)
\end{definition}
\begin{definition}
    We can call a topological space \textit{Hausdorff} if and only if $$\forall x, y \in X x \neq y, \exists U_x, U_y \in \tau; U_x \cap U_y = \phi$$
    Essentially, when you consider two distinct points in a space, then you can also find neighborhood's around those points that do not overlap. 
\end{definition}
\begin{definition}
    A \textit{basis} for a topology $B \subset \tau$ is a set such that each open set in $\tau$ is a union of elements of $B$. The topological space $(X, \tau)$ is considered to be \textit{second-countable} if there is a countable basis for $\tau$.
\end{definition}
\section{Differentiable Manifolds}
We now proceed to figure out what manifolds are (FINALLY)
\begin{definition}
    A \textit{topological manifold of dimension m} is a topological space which is Hausdorff and second-countable, AND locally homeomorphic to $\mathbb{R}^m$. Here a homeomorphism is a mapping that is continuous, bijective, and whose inverse is also continuous.
\end{definition}
\begin{remark}
    When we mean locally homeomorphic, we mean there is an open neighborhood $U \subset M$ and a homeomorphism $\phi: U \to V \subset \mathbb{R}^m$ such that $V$ is an open subset of  $\mathbb{R}^m$
\end{remark}
Some other remarks will also need to be made, to better understand this preliminary concept, as well as lead to other observations that will require proofs
\begin{remark}
    We consider $(U, \phi)$ to be a \textit{chart}. 
\end{remark}
\begin{remark}
    Any subset of a Hausdorff subspace is Hausdorff. All metric spaces are Hausdorff. The case is similar for the case of second-countability.
\end{remark}
But the notion of "charts" means there could exist multiple homeomorphisms between the topological space and $\mathbb{R}^m$, when could both of these be compared? Or rather how can such similar maps be compared in the first place?
\begin{definition}
    We call 2 charts, $(U_1, \phi_1)$ and $(U_2, \phi_2)$ to be smoothly compatible if the following function (with the given restricted domain) is a diffeomorphism.
    $$\phi_2 \circ (\phi_1^{-1})|_{\phi_1 (U_1 \cap U_2)}: \phi_1(U_1 \cap U_2) \to \phi_2(U_1 \cap U_2)$$
    A \textit{diffeomorphism} is a map that is differentiable, bijective, and whose inverse is also differentiable.
\end{definition}
Now that we have a notion of a chart, and the notion of a manifold, what can we do about a collection of charts from a topological space?
\begin{definition}
    We define a \textit{smooth atlas} $\mathcal{A}$ for $M$ to be a collection of charts $(U_i\phi_I), i \in I$ such that they cover the whole of $M$ and that each the charts taken pairwise are smoothly compatible.
\end{definition}
\begin{remark}
    A smooth atlas $\mathcal{A}$ is said to be maximal, if for any other atlas $\mathcal{B}$ that exists such that $\mathcal{A} \subset \mathcal{B} \implies \mathcal{A} = \mathcal{B}$
\end{remark}
Now
\begin{definition}
    A \textit{differentiable or smooth} manifold, is a topological manifold equipped with a maximal smooth atlas.
\end{definition}
\section{Differentiable Maps}
    Let $M$ be a differentiable/smooth manifold (i.e. a topological space with a maximal smooth atlas, i.e. a topological space equipped with a collection of charts, which are pairwise smoothly compatible, i.e. a topological space that is hausdorff and second-countable, that has an an open cover, each element of which is locally homeomorphic to real euclidean space.)
\begin{definition}
    Now we say a mapping $f:M \to \mathbb{R}$ is differentiable at $p \in M \iff \exists$ a chart $(U, \phi)$ aroudn $p$ such that $f \circ \phi^{-1}: \phi(U) \to \mathbb{R}$ is differentiable at $\phi(p)$
\end{definition}
\begin{remark}
    This differentiability, if it holds for one chart, holds for all other charts given the same domain.
\end{remark}
\begin{remark}
    $f: M \to N$ where $M, N$ are both differentiable manifolds, is a differentiable homeomorphism (NOTE: this does not mean it is a diffeomorphism, because $f^{-1}$ may in fact not be differentiable.)
\end{remark}
\end{document}