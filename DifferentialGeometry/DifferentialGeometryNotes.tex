\documentclass[12pt]{book}
\date{\today}
\title{Differential Geometry 1: G0B08a}
\author{Lael John}
\usepackage{amsmath, amsfonts, amssymb, amsthm}

\newtheorem{theorem}{Theorem}[chapter]
\newtheorem{lemma}[theorem]{Lemma}
\newtheorem{corollary}{Corollary}[theorem]
\newtheorem{proposition}{Proposition}[section]

\theoremstyle{definition}
\newtheorem*{definition}{Definition}
\newtheorem{example}{Example}[chapter]
\newtheorem*{huh}{Thoughts}
\newtheorem*{remark}{Remark}
\begin{document}
\maketitle
\chapter*{Preface}
This course seems to involve playing around with manifolds (which I'm given to understand are surfaces not situated \textit{within} another space, but rather are the space of study themselves). In particular, this course seems to involve a lot of geometry (DUH) and building up a calculus on various weird shapes (so to speak). Over the course of the semester, we will see how my initial impressions change.
\tableofcontents
\chapter{Differentiable Manifolds}
This chapter is supposed to cover all the basics that I will end up using this semester in my study of these strange things called \textit{manifolds}, so we build up the required theory, starting from topological spaces, then on to topological manifolds, and then mapping, charting and atlas-ing our way to differentiable manifolds.
\section{Topological Spaces and Manifolds}
This just rehashes now what we've seen in all previous lectures (and is supposed to be prerequisite material) but once more unto the breach
\begin{definition}
    A \textit{topological space} $(X, \tau)$ is when you are given a set $X$ and $\tau \subset P(X)$ such that the following axioms hold true \begin{enumerate}
        \item $\phi, X \in \tau$
        \item For any set $\{U_i\}_{i \in I}$ of $U_i \in \tau \forall i \in I$, then $$ \bigcup_{i \in I} U_i \in \tau$$ where $I$ is an arbitrary index set. 
        \item For a set of $\{U_i\}_{i = 1}^n$, again, where each element is in $\tau$, then $$\bigcap_{i = 1}^n U_i \in \tau$$
    \end{enumerate}
    The elements of $\tau$ are called \textit{open} sets. 
    \begin{huh}Note that you don't really need to define an explicit "finite intersection closure", you could just work with intersection, and then proceed by induction to prove that it works for a finite number of sets.\end{huh}
\end{definition}
\begin{definition}
    Let $p \in X$. Then a \textit{neighborhood} of $p$ is an open subset $U \in \tau$ such that $p \in U \in \tau$
\end{definition}
Finally we think about subsets in a topological space, and whether these arbitrary subsets themselves have a topological structure. It turns out that they do.
\begin{definition}
    If $Y \subset X$, then $(Y, \tau_Y)$ is a topological space where $$\tau_Y = \{ U \cap Y | U \in \tau\}$$ This topology is called an induced topology (intersection of the parents open sets with the subset under scrutiny.)
\end{definition}
We know pause and recall a certain fact about equivalence relations partition the sets they are used on.
\begin{definition}
    Let $\sim$ be an equivalence relation on a topological space $(X, \tau)$ (going to ommit writing the $\tau$ in future). Now consider $\pi: X \to X/\sim$ being a function. Here, since we have a function from a set to it's quotient set (partitioned into equivalence classes by an equivalence relation), we can say $X/\sim$ is a topologial space, by saying $U \in X/\sim$ open whenever $\pi^{-1}(U)$ open in $X$. This definition also implicitly makes $\pi$ a continuous function (because by definition, it pulls back open sets in the target to open sets in the domain.)
\end{definition}
\begin{definition}
    We can call a topological space \textit{Hausdorff} if and only if $$\forall x, y \in X x \neq y, \exists U_x, U_y \in \tau; U_x \cap U_y = \phi$$
    Essentially, when you consider two distinct points in a space, then you can also find neighborhood's around those points that do not overlap. 
\end{definition}
\begin{definition}
    A \textit{basis} for a topology $B \subset \tau$ is a set such that each open set in $\tau$ is a union of elements of $B$. The topological space $(X, \tau)$ is considered to be \textit{second-countable} if there is a countable basis for $\tau$.
\end{definition}
We now proceed to figure out what manifolds are (FINALLY)
\begin{definition}
    A \textit{topological manifold of dimension m} is a topological space which is Hausdorff and second-countable, AND locally homeomorphic to $\mathbb{R}^m$. Here a homeomorphism is a mapping that is continuous, bijective, and whose inverse is also continuous.
\end{definition}
\begin{remark}
    When we mean locally homeomorphic, we mean there is an open neighborhood $U \subset M$ and a homeomorphism $\phi: U \to V \subset \mathbb{R}^m$ such that $V$ is an open subset of  $\mathbb{R}^m$
\end{remark}
Some other remarks will also need to be made, to better understand this preliminary concept, as well as lead to other observations that will require proofs
\begin{remark}
    We consider $(U, \phi)$ to be a \textit{chart}. 
\end{remark}
\begin{remark}
    Any subset of a Hausdorff subspace is Hausdorff. All metric spaces are Hausdorff. The case is similar for the case of second-countability.
\end{remark}
But the notion of "charts" means there could exist multiple homeomorphisms between the topological space and $\mathbb{R}^m$, when could both of these be compared? Or rather how can such similar maps be compared in the first place?
\begin{definition}
    We call 2 charts, $(U_1, \phi_1)$ and $(U_2, \phi_2)$ to be smoothly compatible if the following function (with the given restricted domain) is a diffeomorphism.
    $$\phi_2 \circ (\phi_1^{-1})|_{\phi_1 (U_1 \cap U_2)}: \phi_1(U_1 \cap U_2) \to \phi_2(U_1 \cap U_2)$$
    A \textit{diffeomorphism} is a map that is differentiable, bijective, and whose inverse is also differentiable.
\end{definition}
Now that we have a notion of a chart, and the notion of a manifold, what can we do about a collection of charts from a topological space?
\begin{definition}
    We define a \textit{smooth atlas} $\mathcal{A}$ for $M$ to be a collection of charts $(U_i\phi_I), i \in I$ such that they cover the whole of $M$ and that each the charts taken pairwise are smoothly compatible.
\end{definition}
\begin{remark}
    A smooth atlas $\mathcal{A}$ is said to be maximal, if for any other atlas $\mathcal{B}$ that exists such that $\mathcal{A} \subset \mathcal{B} \implies \mathcal{A} = \mathcal{B}$
\end{remark}
Now
\begin{definition}
    A \textit{differentiable or smooth} manifold, is a topological manifold equipped with a maximal smooth atlas.
\end{definition}
\section{Differentiable Maps}
    Let $M$ be a differentiable/smooth manifold (i.e. a topological space with a maximal smooth atlas, i.e. a topological space equipped with a collection of charts, which are pairwise smoothly compatible, i.e. a topological space that is hausdorff and second-countable, that has an an open cover, each element of which is locally homeomorphic to real euclidean space.)
\begin{definition}
    Now we say a mapping $f:M \to \mathbb{R}$ is differentiable at $p \in M \iff \exists$ a chart $(U, \phi)$ aroudn $p$ such that $f \circ \phi^{-1}: \phi(U) \to \mathbb{R}$ is differentiable at $\phi(p)$
\end{definition}
\begin{definition}
    We now define a function/mapping between manifolds to be differentiable. $f: M \to N$ is smooth at $p \in M$ if \begin{itemize}
        \item $f$ is continuous
        \item there are charts $(U_m, \phi_m)$ and $(U_n, \phi_n)$ around $p, f(p)$ respectively such that $\phi_n \circ f \phi_m^{-1}$ is differentiable at $\phi_m(p)$
    \end{itemize}
\end{definition}
\begin{remark}
    This differentiability, if it holds for one chart, holds for all other charts given the same domain.
\end{remark}
\begin{remark}
    $f: M \to N$ where $M, N$ are both differentiable manifolds, is a differentiable homeomorphism (NOTE: this does not mean it is a diffeomorphism, because $f^{-1}$ may in fact not be differentiable.)
\end{remark}
\section{Partitions of Unity}
What on earth are these things, and why are they being introduced?
\begin{definition}
    Let $M$ be a manifold. Then a \textit{partition of unity} is a family of smooth functions $\{\mathcal{C}_a\}_{a \in A}, \mathcal{C_a}: M \to [0, 1]$ such that \begin{itemize}
        \item $\forall p \in M, \exists U$ a neighborhood of $p; \{a \in A| \mathcal{C}_a |_U \neq 0$ is finite \} (only finitely many non-zero maps inside the particular family, from a neighborhood around a point on the manifold (in the manifold?))
        \item $\sum_{a \in A} \mathcal{C}_a = 1$
    \end{itemize}
\end{definition}
\begin{remark}
    This concept does give us a pretty interesting visualisation, of a family of functions that peaks finitely around a point, before gradually subsiding away from that point?
\end{remark}
\begin{definition}
    If we consider an open cover of $M$, then a partition of unity is said to be \textit{subordinate} to the cover, if the support of $\mathcal{C}_a \subset U_a$. Here we define the support to be the closure of the set of points that the function maps to non-zero reals.
\end{definition}
\begin{remark}
    As a remark, it is possible to show that for every open cover of $M$, there will exist a partition of unity. The proof this statement involves the use of the fact that $M$ is second-countable.
\end{remark}
\section{Submanifolds}
We now come to the last section of this chapter, where we deal with smaller sections of the larger manifolds we've begun to construct.
\begin{definition}
    Let $N$ be a smooth manifold of dimension $n$. Then $M \subset N$ is a submanifold of dimension $m \iff$ 
    $$\forall p \in M \exists (U, \phi), (p \in U); \phi (U \cap M) = \phi(U) \cap \mathbb{R}^m\times\{0\}$$ This is what we call a chart that has been adapted to $M$. Such a set inherits the structure of a manifold, with its atlas given by the correspondingly restricted charts from the parent manifold.
\end{definition}
\begin{remark}
    If $M \subset N$ is a submanifold, then by definition it is a manifold of dimension $m$. The smoothness of the parent manifold is inherited by all its submanifolds.
\end{remark}
\chapter{Tangent Vectors}
Before I begin writing this section, the notion of having a \textit{tangent} seems pretty straightforward, having a line/plane/hyperplane intersecting a surface/space at exactly one point. Since I've seen some applications of tangents in calculus (through the value of the derivative at a point being the measurement of the slope of the tangent vector to the function at that particular point) It may be interesting to see how tangent \textit{vectors} and \textit{spaces} begin to play a role in understanding how smooth manifolds behave. In particular, I would've thought previously that the "smoothness" of the manifold is a result of the derivative of the function generating said surface containing the space. (As I know now, differentiable manifolds don't really reside \textbf{within} a space, they simply are the space.)\\
Also an observation to make given what I currently understand of Einsteins theory of how gravity is simply a curvature of space time, this makes it seem like the whole of our observable universe is simply a smooth manifold, and its curvature could have something to do with how its tangent vectors and spaces behave?\newpage
\section{Tangent vectors and spaces}
\begin{definition}
    We begin with the definition of a tangent vector $p \in M$, where $M$ is a manifold. It is defined to be the equivalence class $[\gamma]$ of smooth curves. \\We define these to be $\gamma: (-\epsilon, \epsilon) \to M); \gamma(0) = p$ and $\gamma_1 \sim \gamma_2 \iff \exists$ a chart $(U, \phi); p \in U$ such that $(\phi \circ \gamma_1)' (0) = (\phi \circ \gamma_2)'(0)$
\end{definition}
\begin{remark}
    Just to reiterate, these tangent vectors are the class of all curves passing through a particular point on the manifold, such that they pass through the point when the parameter is 0. When we consider 2 curves to be represented by the same class, that means that their when we look at them through a chart around the point in consideration, their derivatives in $\mathbb{R}^m$ are equal at the point.
\end{remark}
\begin{remark}
    The second remark to be made is that as long as you have a tangent vector, then it remains a tangent vector regardless of the chart being used. The definition merely requires one chart. We can think of this by seeing that if $\psi$ is a chart different from $\phi$, then $\psi \circ \phi^{-1}$ is a chart that allows for the relation to still hold.
\end{remark}
Because we can talk about tangent \textit{vectors}, this gives us a hint towards figuring out what kind of algebraic structure we can impose on these vectors/how they interact with each other.
\begin{definition}
    A \textit{tangent space} is defined at a point $p \in M$, as the set of all tangent vectors at $p$. This space is denoted by $T_pM$.
\end{definition}
\begin{proposition}
    $T_pM$ is a vector space, for some point $p \in M$
\end{proposition}
\begin{proof}
    Simply consider an arbitrary chart $p \in (U, \phi)$. Then we can obtain a function from $T_pM \to \mathbb{R}^m$ defined by $[\gamma] = (\phi \circ \gamma)'(0)$. Because such a function is well defined (each vector corresponds to a point in the target, vectors that are equal have the same image) and that the function is injective (by definition of the equivalence class). It remains to show surjectivity. Taking an arbitrary vector $v \in \mathbb{R}^m$, and setting considering the line $\phi(p) + tv$, we get a curve on our manifold, $\gamma(t) = \phi^{-1}(\phi(p) + tv)$. Such a $\gamma$ definitely maps to $v$, thus we have a bijection between a neigborhood of tangent space, and a local neighborhood of $\phi(p) \in \mathbb{R}^m$  
\end{proof}
\begin{definition}
    Let $p \in M$, $(U, \phi)$ a chart such that $p \in U$. Then we can denote the components of $\phi(p) = (x_1, x_2 \ldots x_m)$ where each $x_i: U \to \mathbb{R}$. Then when we consider the isomorphism given by the function taking $T_pM \to \mathbb{R}^m, [\gamma] \mapsto (\phi \circ \gamma)'(0)$, then we can see that the canonical basis of $\mathbb{R}^m$ induces a basis on $T_pM$. This basis can be given by $$\{\frac{\partial}{\partial x_1}|_p, \frac{\partial}{\partial x_2}|_p \ldots \frac{\partial}{\partial x_m}|_p\}$$
\end{definition}
\section{Derivative of a Map}
We now consider how to define the derivative of a smooth map (as we recall, a map between manifolds, continuous, and whose composition with a chart from one and the inverse of a chart from the other is differentiable at a point). 
\begin{definition}
    The derivative of a smooth map $f: M \to N$ at $p \in M$ is given by $(f_*)_p: T_pM \to T_(f_p)N$ where $[\gamma] \mapsto [f \circ \gamma]$
\end{definition}
\begin{proposition}
    $(f_*)_p$ is well defined and linear.
\end{proposition}
\begin{proof}
    This should be pretty easy to check. For well defined-ness, consider a tangent vector $[\gamma]$. Then $\gamma(0) = p$. when we consider $(f_*)_p([\gamma]) = [f \circ \gamma]$, then we need to show that this is also a tangent vector and $f \circ \gamma(0) = f(p)$. \\This is true by definition, therefore every element in the domain has an image in the codomain.\\ If we consider $[\gamma_1] = [\gamma_2]$. Then $ (f_*)_p([\gamma_1]) = [f \circ \gamma_1] = [f \circ \gamma_2] = (f_*)_p([\gamma_2])$ (Here we're simply interchanging the classes because nominally, they each represent the same set.) Thus each element in the domain is mapped to a single image.\\ Now to check for linearity, if we consider $(f_*)_p(\alpha[\gamma_1] + \beta[\gamma_2])$ where $\alpha, \beta$ are reals, then trivially this is the same as $\alpha[f \circ \gamma_1] + \beta[f \circ \gamma_2]$\\ (I say trivially, but really multiplying by a scalar real just expands/shrinks the curve in $M$ and thus also correspondingly expands/shrinks the curve in $N$. It becomes possible to add the curves because they are both 0 as they pass through $p \in M$.)
\end{proof}
\begin{remark}
    We could also prove this very simply by considering the fact that the charts are linear, and that the smooth map is also linear and well defined. Since we are then performing a simple composition of functions, along with the differentiation operator that is also linear, then the final mapping is also well defined and linear.
\end{remark}
\begin{remark}
    Our second remark more naturally follows, that the chain rule can be applied for a composition of smooth maps. \\ As an example, if $g, f$ are smooth maps between spaces $M, N, L$, then $\forall p \in M$, we have $(g \circ f)_{*p} = (g_*)_{f(p)}(f_*)_p$ which maps $T_pM \to T_{g(f(p))}L$
\end{remark}
\begin{remark}
    One final remark to be made is that if we consider a chart $(U, \phi)$ such that $p \in U \subset M$, then $\phi$ is a diffeomorphism of manifolds.
\end{remark}

\section{Regular-Level Set Theorem}
We begin by revising (or hearing for the first time) the inverse function theorem in $\mathbb{R}^n$. If we consider a diffeomorphism between 2 open subsets of $\mathbb{R}^n$, then for all elements of the domain, the diffeomorphism becomes an isomorphism, because it also has an inverse.
\begin{lemma}
    Let $U \subset \mathbb{R}^n$ be open. Let the function $f: U \to \mathbb{R}^n$ be smooth such that $D_q f: \mathbb{R}^n \to \mathbb{R}^n$ is an isomorphism for some $q \in U$. Then there exists $V \subset U$, of $q$ such that $f|_V: V \to f(V)$ is a diffeomorphism.
\end{lemma}
Because we know that the above stated inverse function theorem is true in n-dimensional real space, we can try and port such a theorem over to our study of manifolds, resulting in the following corollary.
\begin{corollary}
    Let $f: M \to N$ be a smooth map, $p \in M$ such that $(f_*)_p: T_pM \to T_{f(p)}N$ an isomorphism. Then $\exists W \subset M, p \in W$ such that $f|_W: W \to f(W)$ is a diffeomorphism.
\end{corollary}
\begin{remark}
    A rough sketch of the proof simply involves taking charts on $M, N$ and applying the inverse function theorem in $\mathbb{R}^m, \mathbb{R}^n$ and then composing the charts with the function in consideration.
\end{remark}
Now that we have such a foundation, proving the existence of a diffeomorphism when dealing with a smooth map and an isomorphism, we can now think of submersions (so to speak) and regular values.
\begin{lemma}
    Let $U \in \mathbb{R}^n \times \mathbb{R}^{k-n}$ such that $ 0 \in U$. Also consider $f: U \to R$ to be a smooth function, such that $(D_0 f)|_{\mathbb{R}^n \times \{0\}}: \mathbb{R}^n \times \{0\} \to \mathbb{R}^n$ is an isomorphism. Then, there exists a diffeomorphism $\tau$ between neighborhoods in $\mathbb{R}^n \times \mathbb{R}^{k-n}$ such that $f \circ \tau^{-1} = \pi$
\end{lemma}
The previous \textit{Submersion theorem in $\mathbb{R}^n$} basically tells us that when we consider an diffeomorphism from an open set containing the origin in $k$ dimensional space, to $n$ dimensional space $k > n$, we can find another diffeomorphism such that when we compose both, we get simply the projection onto $n$ dimensional space.
\begin{definition}
    A \textit{regular value} of $f: M \to N$, a smooth map is a point $c \in N$ such that $\forall p \in f^{-1}(c), (f_*)_p: T_pM \to T_{f(p)}$ is surjective.
\end{definition}
We now move on to our final theorem of the section, building on everything we've learnt about tangent vectors, spaces, derivatives of smooth maps and regular values.
\begin{theorem}
    Let $f: M \to N$ be a smooth map, and let $c \in N$ be a regular value, such that $f^{-1}(c) \neq \emptyset$. Then \begin{itemize}
        \item $f^{-1}(c)$ is a submanifold of $M$, with a dimension of $m - n$ (Here we're assuming the dimension of $M > N$)
        \item $\forall p \in f^{-1}(c): T_p f^{-1}(c) = \textnormal{Ker}(f_*(p))$
    \end{itemize}
\end{theorem}
\section{Tangent vectors as derivations at a point}

\end{document}