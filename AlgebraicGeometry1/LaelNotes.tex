\documentclass[12pt]{book}
\date{\today}
\title{Algebraic Geometry 1: G0A80a}
\author{Lael John}
\usepackage{amsmath, amsfonts, amssymb, amsthm}

\newtheorem{thm}{Theorem}[chapter]
\newtheorem{lem}[thm]{Lemma}
\newtheorem{cor}{Corollary}[thm]
\newtheorem{proposition}{Proposition}[section]

\theoremstyle{definition}
\newtheorem*{defn}{Definition}
\newtheorem{eg}{Example}[chapter]
\newtheorem*{huh}{Thoughts}
\newtheorem*{rem}{Remark}
\begin{document}
\maketitle
\chapter*{Preface}
I am writing this document while taking a class in Algebraic Geometry 1, at KU Leuven in Belgium, 2022. This book will serve the dual purpose of helping me rewrite my course notes to allow me to get better acquainted with the material, as well as offer me the chance to work through exercises on my own, and find and fix flaws in my understanding when they do arise. This course was taught by Nero Buduur. 
\tableofcontents
\chapter{Affine Algebraic Varieties}
What are these strange things? They are essentially the sets of zeros of polynomials, for example
$$f(z) = a_0 + a_1z + a_2z^2 + \ldots a_nz^n; z \in \mathbb{C}$$
These sets are things we've already seen in the past, namely when dealing with conic sections, and other surfaces in 3 dimensions. What we seem to be doing is essentially figuring out what polynomials describe surfaces, and then studying those properties of the polynomials to better understand the surfaces generated themselves. Bear in mind these are n-dimensional surfaces, so while dealing with them visually is much easier in 2 and 3 dimensions, we run into problems very quickly. 

\begin{huh}
Note that the textbook suggests that an algebraic variety is simply a geometric object that \textit{locally resembles} Euclidean space. In essence, these objects are providing necessary characteristics/or they seem to be converting geometric objects into some intermediate state to be studied. 
\end{huh}

\section{Formal Definition}
\begin{defn}[Affine Algebraic Variety]
The common zero set, of a collection $\{F_i\}_{i\in I}$, complex polynomials in complex n dimensional space, namely $\mathbb{C}^n$. We thus write 
$$V = \mathbb{V}(\{F_i\}_{i \in I}) \subset \mathbb{C}^n$$
\end{defn}
\begin{rem} Here it doesnt seem to matter what the index set is, countable or uncountable
\end{rem}
\begin{eg}Consider $V = \mathbb{V}(x_1, x_2) \subset \mathbb{C}^3$. This AAV corresponds to the common zero set of the two polynomials $x_1$ and $x_2$ in $\mathbb{C}^3$. We end up with only the line $x_3 = 0$. 
\begin{huh}Even though these "lines" are supposed to be considered as such, remember they're actually dealing with complex variables, so in reality we're dealing with $\mathbb{R}^6$\end{huh}
\end{eg}
One of the issues with this definition of an AAv, is that we run into the problem that these objects are technically \textit{embedded} in a higher space, i.e. their properties may depend on the space in which they're situated. That's not really ideal.
\begin{rem}
Also remember, we're specifying that these are \textbf{COMPLEX} affine algebraic varieties, because for reasons we'll see later, using $\mathbb{C}$ makes life a lot easier than using other fields like $\mathbb{R}$ or $\mathbb{Q}$.
\end{rem}
\begin{eg}
Another set of examples to consider are the n-dimensional complex space $\mathbb{C}^n$, along with the empty set, and singleton sets (i.e. single point sets). All these are trivial examples of affine algebraic varieties (???).\\
Okay so, the space $\mathbb{C}^n$ is an AAV, because $$\mathbb{C}^n = \mathbb{V}(0)$$
which makes sense because its vacuously true. Similarly for $\phi$, the polynomial required is any non-zero constant polynomial.\\
Finally for single point sets, the only way that point is the only solution to a polynomial, is if the following holds
$$(a_1, a_2, \ldots a_n) = \mathbb{V}((x-a_1)(x-a_2)\ldots(x-a_n))$$
\end{eg}
\begin{eg}
Consider another example, the zero set of one \textit{convex} polynomial in $\mathbb{C}^2$, which we'll be projecting into $\mathbb{R}^2$ as a parabola for now. Consider $$V_1 = \mathbb{V}(y - x^2) \subset \mathbb{C}^2$$ or 
$$V_2 = \mathbb{V}(x^2y + xy^2 - x^4 - y^4) \subset \mathbb{C}^2$$ or even
$$V_3 = \mathbb{V}(y^2 - x^2 - x^3) \subset \mathbb{C}^2$$
\begin{huh}
Insert the graphs here from $R^2$ later
\end{huh}
\end{eg}
\begin{eg}
Now also, the zero set of a single polynomial in an arbitrary dimension becomes a \textit{hypersurface} in $\mathbb{C}^n$. One classic example is $$V = \mathbb{V}(x^2 + y^2 - z^2) \subset \mathbb{C}^3$$ which represents a cone in regular $\mathbb{R}^3$, but lies in $\mathbb{C}^3 \approx \mathbb{R}^6$
\begin{rem}
Note that in the actual variety, the waist of the cone does not become a single point, as it's representation in regular 3d space would suggest.
\end{rem}
\end{eg}
\begin{eg}
We can now look at the zero set of linear polynomials. These too form affine algebraic varieties, called affine \textit{hyperplanes}. These could be, for example, the line $ax + by = c$ defined in the complex plane $\mathbb{C}^2$, with $a, b, c$ complex scalars.
Now a \textit{linear affine algebraic variety} is essentially the common zero set, of a collection of linear polynomials, of similar form. Example: $$a_1x_1 + a_2x_2 + \ldots + a_nx_n - b$$ Now each such polynomial seems to generate a line (linear get it?) in $\mathbb{C}^n$, so then when you combine some $k$ linearly independent polynomials like this (meaning none of them can be expressed as a linear combination of the other polynomials), you get a complex space of dimension $n-k$
\begin{rem}
Think carefully here, given lets say two lines in R3 that are \textit{linearly independent}, that simply means that one is not a multiple of the other, i.e. they do not fall on the same line through the origin. That's 3 vectors pointing in non linear directions. those lines are perpendicular to the planes generated by the 3 non linear vectors. Since they're non linear, their planes must intersect (is there a theorem for this) and give rise to a line (not a plane, definitely not a point, again look for a theorem) and thus, WHEN IN 3D space, if you have 2 linear polynomials, YOU GET A LINE (1 dimensional) N-K.
\end{rem}\begin{huh}
Whats the difference between hypersurface and hyperplane?
\end{huh}
\end{eg}
\begin{eg}
Now we consider the set of all $n \times n$ matrices, which technically belong to the space $\mathbb{C}^{n^2}$. In this new, weirdly high dimensional space, consider $SL(n, \mathbb{C})$ or the space of all complex $n\times n$ matrices with determinant 1. These form an affine algebraic variety when you consider their characteristic polynomials. This leads to a singular hypersurface (when considering only one matrix) or a hyperplane (again, need to figure out the difference)
\end{eg}
\begin{eg}
    A final example is when we consider, in the same space as the last example, the set of all matrices in $\mathbb{C}^{n^2}$ with rank at most $k$. This example, called the \textit{determinantal variety}, is the whole of $\mathbb{C}^n$ if $k \geq n$ 
    \begin{rem}
        This example requires a lot more thought before I consider it understood.
    \end{rem}
\end{eg}
\section*{Counter/Nonexamples}
\begin{enumerate}
    \item An open ball in $\mathbb{C}^n$, endowed with the regular Euclidean topology is not an algebraic variety. (For a full proof, see exercises). Similarly $GL(n, \mathbb{C})$ and $U(n)$ are also not algebraic varieties.
    \item A closed square in $\mathbb{C}^2$ is a closed set that is not an algebraic variety.
    \item Graphs of transcendental functions are not algebraic varieties (because they cannot be represented by finite polynomials probably.)
\end{enumerate}

\section{The Zariski Topology}
We begin this section with a slight review of topology thankfully. This intuition will then allow us to ask the question of what it means to impose a topology on the set of algebraic varieties. 
\begin{defn}
    A \textit{topology} on a set $X$ is a collection of subsets $\tau \subset P(X)$ such that the following properties hold. \begin{enumerate}
        \item $\phi, X \in \tau$
        \item Let $\{U_i\}_{i \in I}$ be a collection of members of $\tau$. Then $\bigcup_{i \in I} U_i \in \tau$
        \item Let $\{U_i\}_{i = 1}^n$ be a collection of members of $\tau$. Then $\bigcap_{i = 1}^n U_i \in \tau$
    \end{enumerate}
    The elements of $\tau$ are open sets, and $(X, \tau)$ is a topological space.
\end{defn}
\begin{rem}
    It suffices to state that the topology is closed under pairwise intersection, because finite intersection then comes as a result of an inductive process.
\end{rem}
We now begin to consider what it means for a function to be continuous, in topological language, because we will then apply this logic to later concepts.
\begin{defn}
    A map $f: X \to Y$ between two topological spaces is continuous $$\iff f^{-1}(U \in \tau_Y) \in \tau_X$$
\end{defn}
This definition is pretty straight forward, open sets from the target space are \textit{pulled back} to open sets in the domain space.
We can also now define a \textit{homeomorphism} as being $f:X \to^{cts} Y$ such that $f$ is bijective, and that in turn $f^{-1}$ is also continuous.
\begin{huh}
    Interestingly, a topology can also be defined by simply taking the complements of all the properties we initially used, De Morgans Laws helping us out quite a bit. Closure under arbitrary intersection and finite union, when dealing with closed sets.
\end{huh}
\begin{eg}
    We begin with noting that a classic example of a topology in $\mathbb{C}^n = \mathbb{R}^{2n}$ is the Euclidean topology of open balls (generated by taking distances componentwise)
\end{eg}
\begin{rem}
    We pause here to note that any affine algebraic variety in $\mathbb{C}^n$ is closed in the Euclidean topology, and provide a proof below as follows.
    \begin{proof}
        $$V = \mathbb{V}(\{F_i\}_{i \in I}) = \bigcap_{i \in I} \mathbb{V}(F_i)$$
        This fact being a little obvious after taking a second, we proceed by looking at a particular polynomial function from the set taken above $F_i: \mathbb{C}^n \to^{cts} \mathbb{C}$.\\
        Now the crux of the proof. Since $0$ is closed in $\mathbb{C}$ (singleton sets are their own closures in any space), therefore, when we look at the preimage of $0$ under $F_i$, we realise that the preimage must necessarily also be closed. Thus we see that $F_i^{-1} (0)$ is closed in $\mathbb{C}^n$.\\
        Clearly this must also mean this fact is true for all $i \in I$, thus, 
        $$\bigcap_{i \in I} F_i^{-1} (0) $$
        is closed in $\mathbb{C}^n$. BUT this just means that  $$\bigcap_{i \in I} \mathbb{V}(F_i) $$
        is closed in $\mathbb{C}^n$.\\
        Thus $V$ is closed.
    \end{proof}
\end{rem}
\begin{huh}
    This allows us to see how open balls in $\mathbb{C}^n$ are not affine algebraic varieties. The closed boxes though still elude us.
\end{huh}
Now we can begin to proceed with our construction of a topology on affine algebraic varieties in $\mathbb{C}^n$
\begin{lem}
    The intersection of any affine algebraic varieties, and the union of finitely many affine algebraic varieties are affine algebraic varieties.
\end{lem}
\begin{proof}
    We begin with the first part, dealing with the intersection of an arbitrary number of affine varieties. We look at $$\bigcap_{k \in K} \mathbb{V}(\{F_i\}_{i \in I_k}) = \bigcap_{k \in K} \bigcap_{i \in I_k} \mathbb{V}(F_i) = \mathbb{V}(\{F_i\}_{i \in I_k, k \in K}) = \mathbb{V}(\{F_i\}_{i \in \cup_{k \in K} I_k}) $$
    This pretty straightforward series of steps simply assumes an arbitrary number of affine algebraic varieties, writes each as an intersection of hypersurfaces, and then reframes the intersection as being an affine algebraic variety of the combined family of polynomials being considered.\\
    Now when we consider finite union, we begin with two affine algebraic varieties, $V_1 = \mathbb{V}(\{F_i\}_{i \in I})$ and $V_2 = \mathbb{V}(\{F_j\}_{j \in J})$. The union between these two, $V_1 \cup V_2$ evaluates to $$\bigcap_{i \in I, j \in J} (\mathbb{V}(F_i) \cup \mathbb{V}(F_j))$$ (using the distributive property of intersection over unions). We can simplify the inner union by simply rewriting it as a hypersurface generated by the product of the two polynomials, i.e. $$ \bigcap_{i \in I, j \in J} \mathbb{V}(F_iF_j)$$
    Now we can rewrite this as being an affine algebraic variety of the following form $$ \mathbb{V}(\{F_iF_j\}_{i \in I, j \in J})$$ 
\end{proof}

\begin{defn}
    The topology on $\mathbb{C}^n$ with closed subsets defined by affine algebraic varieties is the \textit{Zariski} Topology.
\end{defn}
\begin{rem}
    Now because there may arise confusion, we denote the vector space $\mathbb{C}^n$ as it is, but the topological space with the Zariski topology we denote with $\mathbb{A}^n$ denoting \textit{affine n-space}. Quick asides here being that this topology is not Hasudorff, and it doesn't make sense to measure distances in this space. We are only concerned with algebraic manipulations in this space.
\end{rem}
\begin{rem}
    We also note that though every affine algebraic variety is closed in the Euclidean topology, not every euclidean closed set is zariski-closed. This is due to the property of the Euclidean topolgy being generated by open balls of arbitrarily small radius, though by necessity, zariski-open sets are quite large. However, it does make sense to talk of a zariski topology in fields other than the complex numbers or the reals. 
\end{rem}
\begin{defn}
    Every affine algebraic variety inherits a topology from the ambient $\mathbb{A}^n$ space, namely the Zariski topology that is induced on this smaller subset. In particular now, closed sets in $V$ are $V \cap W$ where $W \subset \mathbb{A}^n$ Thus closed sets of $V$ under this topology become affine algebraic \textit{subvarieties}
\end{defn}

\section{Morphisms}
Seems like a familiar sounding word, though what are these things. They seem to be mappings from one affine algebraic variety to another. 
\end{document}